% -*- coding: utf-8 -*-
% !TEX program = xelatex
\documentclass{jnuthesis}
\usepackage{amsmath}
\usepackage{algorithmic}
\usepackage{array}
\usepackage{fixltx2e}
\usepackage{stfloats}
\usepackage{url}
\usepackage{multicol}
\usepackage{graphicx}
\usepackage{ctex}
\usepackage{subfigure}
\usepackage{float}
\usepackage{indentfirst}
\usepackage{booktabs}
\usepackage{lscape}
\usepackage{caption}
\usepackage{xcolor}  
\usepackage{tikz}  
\usepackage{tikz,mathpazo}
\usetikzlibrary{shapes.geometric, arrows}
\usetikzlibrary{arrows,shapes,chains}  
\usepackage{amsthm,amsmath,amssymb}
\usepackage{amsfonts}
\usepackage{tikz}
\usetikzlibrary{positioning, shapes.geometric}
\usetikzlibrary{calc}

\tikzstyle{format}=[rectangle,draw,thin,fill=white]  
%定义语句块的颜色,形状和边
\tikzstyle{test}=[diamond,aspect=2,draw,thin]  
%定义条件块的形状,颜色
\tikzstyle{point}=[coordinate,on grid,]  
\tikzstyle{startstop} = [rectangle, rounded corners,minimum height=0.7cm,text centered, draw=black, fill=red!30]
\tikzstyle{io} = [trapezium, trapezium left angle=70, trapezium right angle=110,  minimum height=0.7cm, text centered, draw=black, fill=blue!30]
\tikzstyle{process} = [rectangle,  minimum height=0.7cm, text centered, draw=black, fill=orange!30]
\tikzstyle{decision} = [diamond, minimum height=0.7cm, text centered, draw=black, fill=green!30]
\tikzstyle{arrow} = [thick,->,>=stealth]




\usepackage{subfigure}
%\usepackage[backend=biber,style=gb7714-2015]{biblatex}
%\usepackage[super]{gbt7714}
\usepackage{gbt7714}
\usepackage{titlesec}\titleclass{\chapter}{straight}
\titleformat{\chapter}[hang]  {\huge\bfseries}{\arabic{chapter}}{1em}{}

\usepackage{listings}
\usepackage{xcolor} 
\makeatletter
\renewcommand{\maketag@@@}[1]{\hbox{\m@th\normalsize\normalfont#1}}%
\makeatother

\definecolor{mygreen}{rgb}{0,0.6,0}
\definecolor{mygray}{rgb}{0.5,0.5,0.5}
\definecolor{mymauve}{rgb}{0.58,0,0.82}
\lstset{ %
	backgroundcolor=\color{white},   % choose the background color
	basicstyle=\footnotesize\ttfamily,        % size of fonts used for the code
	columns=fullflexible,
	breaklines=true,                 % automatic line breaking only at whitespace
	captionpos=b,                    % sets the caption-position to bottom
	tabsize=4,
	commentstyle=\color{mygreen},    % comment style
	escapeinside={\%*}{*)},          % if you want to add LaTeX within your code
	keywordstyle=\color{blue},       % keyword style
	stringstyle=\color{mymauve}\ttfamily,     % string literal style
	frame=shadowbox,
	rulesepcolor=\color{red!20!green!20!blue!20},
	% identifierstyle=\color{red},
	numbers=left, 
	numberstyle=\tiny,
	% escapeinside=' ',
	xleftmargin=2em,
	xrightmargin=2em, 
	aboveskip=1em
}

\begin{document}

\renewcommand{\title}{使用说明} % 英文标题
\renewcommand{\biaoti}{使用说明}  % 中文标题
\renewcommand{\xueyuan}{国际能源学院}
\renewcommand{\zhuanye}{电气工程及其自动化}
\renewcommand{\xingming}{余思贤}
\renewcommand{\xuehao}{2018054439}
\renewcommand{\daoshi}{莫维科}

\chapter*{exe文件在dist文件夹中}
\chapter{命令行调用}
\begin{lstlisting} 
	scihub.exe [-p=A] [-rn=B] [-st=C] [-rt=D] [-i=E] [-sci=F] [-m=G]
\end{lstlisting}

\textbf{参数解析:}
\begin{enumerate}
	\item p:存放doi文件路径,默认为scihub.exe同路径下的doi.exe
	\item rn:是否使用错误重试,默认是True,接受参数为0或1,0为False,1为True
	\item st:错误重试等待时间,默认30秒
	\item rt:最大重试次数,默认5次
	\item i:是否使用idm进行下载,若传入为False则使用原本的下载方式。默认True,接受参数为0或1,0为False,1为True
	\item sci:选择link.txt文件中第几条scihub连接。默认为1(从0开始)
	\item m:论文命名规则,默认使用doi命名。接受参数为0或1,0为使用文章标题命名,1为使用doi命名。
\end{enumerate}

\textbf{dome:}
\begin{lstlisting}
	scihub.exe --i=0 --rn=0 --p=C:\Users\VerNe\Desktop\新建文件夹\SciDownl-AutoCaptcha-master\dist\doi.txt
\end{lstlisting}

此命令为不使用idm,不使用错误重试,传入doi路径为C:\textbackslash Users\textbackslash V\\erNe\textbackslash Desktop\textbackslash 新建文件夹\textbackslash SciDownl-AutoCaptcha-master\textbackslash dist\textbackslash doi.txt

\textbf{p的参数可以直接将文件拖入终端}

\chapter{重新打包为exe}

使用pyinstaller进行打包,在终端输入如下命令为:
\begin{lstlisting} 
	pyinstaller -F scihub.py
\end{lstlisting}

需要安装pyinstaller,安装命令:
\begin{lstlisting} 
	pip install pyinstaller
\end{lstlisting}

完成之后文件会生成在dist文件夹中。

\textbf{建议新建ven仅安装需要的库,不然打包出来会非常非常大。}
\end{document}
